\newpage
\chapter{Functors}
\section{Introduction}
So far we've encountered a lot of the type classes that are present inside Prelude. We've seem how \texttt{Eq} represents the types that can be equated and \texttt{Ord} the ones that can be ordered. We've used \texttt{Show} to classify the types that can be displayed and \texttt{Read} the ones that can be read. \\
An now, we are talking about the \textbf{\texttt{Functor}} type class, which is basically for things that can be mapped over. \\
The first reaction when listening to the keyword "map" is probably thinking about lists, and that is perfect because the list type of part of the \texttt{Functor} class.
\linebreak \linebreak

Let's see now how \texttt{Functor} is implemented.

\begin{lstlisting}[language=haskell]
class Functor f where
  fmap :: (a -> b) -> f a -> f b
\end{lstlisting}

We see that it defines one function, \texttt{fmap}, and doesn't provide any default implementation for it. We see that \texttt{fmap} takes a function from one type to another and a functor applied with one type, and returns a functor applied with another type. \\


\section{Examples}
\subsection{Lists}
In order to make things more clear, let's take a step back and see a similar type, the one of \texttt{map}.

\begin{lstlisting}[language=haskell]
map :: (a -> b) -> [a] -> [b]
\end{lstlisting}

Interesting! It takes a function from one type to another and a list of one type, and returns a list of another type. This is a perfect example of \texttt{Functor}. In fact, \texttt{map} is just a \texttt{fmap} that works only on lists and is defined as follows.

\begin{lstlisting}[language=haskell]
instance Functor [] where
  fmap = map
\end{lstlisting}

Notice that there is no need to write \texttt{instnace Functor [a] where}, because from \texttt{fmap :: (a -> b) -> f a -> f b} we see that \texttt{f} has to be a type constructor that takes one type. \texttt{[a]} is already a concrete type (of a list with any type inside it), while \texttt{[]} is a type constructor that takes one type and can produce types such as \texttt{[Int], [String]} or even \texttt{[[String]]}.
\linebreak \linebreak

Since for lists \texttt{fmap} is just \texttt{map}, we get the same results when using them on lists. 

\begin{lstlisting}[language=haskell]
Prelude> fmap (*2) [1..3]
[2, 4, 6]

Prelude> map (*2) [1..3]
[2, 4, 6]
\end{lstlisting}


\subsection{Maybe}
Similar to \texttt{[]}, also \texttt{Maybe} is a \texttt{Functor}.

\begin{lstlisting}[language=haskell]
instance Functor Maybe where
  fmap f (Just x) = Just(f x)
  fmap f Nothing = Nothing 
\end{lstlisting}


\subsection{Tree}
Another type that can be mapped over is \texttt{Tree a}.

\begin{lstlisting}[language=haskell]
instance Functor Tree where
  fmap f (Leaf a) = Leaf(f a)
  fmap f (Node left x right) = Node (fmap f left) (fmap x) (Node fmap f right)
\end{lstlisting}


\section{Wrapping up}
In a more general way, a \texttt{Functor T a} is a container that contains values of type \texttt{a} and \texttt{fmap f} applies \texttt{f} to each value inside the container. \\
We can see that \texttt{IO} is an instance of \texttt{Functor} too.

\begin{lstlisting}[language=haskell]
instance Functor IO where
  fmap f mx = do x <- mx
                 return (g x)
\end{lstlisting}

The usage of \texttt{fmap} is that it can process the values of any container that is a \texttt{Functor}, and this generalization propagates to functions defined by the mean of \texttt{fmap} itself. 

\begin{lstlisting}[language=haskell]
inc = fmap (+1)

Prelude> inc (Just 1)
Just 2

Prelude> inc [1, 2, 3]
[2, 3, 4]

Prelude> inc (Node (Leaf 1) 0 (Leaf 2))
Node (Leaf 2) 1 (Leaf 3) 
\end{lstlisting}

\section{Functor laws}
In order to define a well behaved \texttt{Functor} we need to make it respect some laws. 

\begin{enumerate}
	\item \texttt{fmap id = id} \\
		  That is, \texttt{fmap} needs to preserve the identify.
		  
	\item \texttt{fmap (f . g) = fmap g . fmap h} \\
		   That is, \texttt{fmap} needs to preserve the function composition.
\end{enumerate}

These rules, together with the type of \texttt{fmap}, assure that \texttt{fmap} is a mapper that does not reorder the values inside the container which is applied to.

\section{Interesting fact}
For any parameterized type inside Haskell there is \texttt{at most one} function \texttt{fmap} that satisfies the laws. So, if we can make a parameterized type into a \texttt{Functor}, then \texttt{fmap} is unique.