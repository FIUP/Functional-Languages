\newpage
\chapter{List comprehension, recursive functions and higher order functions}

\section{List comprehension}
\textbf{List comprehension} is a simple way provided by Haskell to create list using a generator.\\
Let's see an example

\begin{lstlisting}[language=haskell]
Prelude> [ x^2 | x <- [1..5] ]
[1, 4, 9, 16, 25
\end{lstlisting}

The generator used is \texttt{x <- [1..5]} which creates a list containing the values from 1 (inclusive) to 5 (inclusive). The list comprehension than acts as a loop, iterating over all the elements and applying the function which is written on the left side of \texttt{|}, obtaining the final list.
\linebreak \linebreak

Note that when there are multiple generators they acts as nested loops, with the first one written acting as the enclosing loop and the second one as the nested loop.

\begin{lstlisting}[language=haskell]
Prelude> [ (x, y) | x <- [1, 2, 3], y <- [4, 5] ]
[(1, 4), (1, 5), (2, 4), (2, 5), (3, 4), (3, 5) ]

Prelude> [ (x, y) | y <- [4, 5], x <- [1, 2, 3] ]
[(1, 4), (2, 4), (3, 4), (1, 5), (2, 5), (3, 5)]
\end{lstlisting}

It is also possible to use a first generator to determine the values that a second generator should generate.

\begin{lstlisting}[language=haskell]
Prelude> [ (x, y) | x <- [1..3], y <- [x..3] ]
[(1, 1), (1, 2), (1, 3), (2, 2), (2, 3), (3, 3)]
\end{lstlisting}

Also, list comprehension can use guards in order to determine if a value should be put inside the list that is being generated or not.

\begin{lstlisting}[language=haskell]
-- | Gets all the factors of a given number
factors :: Int -> [Int]
factors n = [x | x <- [1..n], n `mod` x == 0]

-- | Tells if a number is a prime number or not 
prime :: Int -> Bool
prime n = factors n == [1, n]

-- | Generates the first n prime numbers
primes :: Int -> [Int]
primes n = [x | x <- [2..n], prime x]
\end{lstlisting}


\section{Recursive functions}
Recursive functions are one of the most powerful tools that Haskell provides, but they can cause a lot of trouble during the development too. In order to define a simple and well typed function easily, here are some advices.

\begin{enumerate}
	\item Define the type of the function
	\item Enumerate the cases
	\item Define the simpler cases
	\item Define the remaining cases
	\item Generalize and simplify 
\end{enumerate}

Let's see how we can apply these steps while writing \texttt{drop}, a function which takes a list as an input and drops the first \texttt{n} elements returning the remaining ones. 

\begin{enumerate}
	\item Define the type\\
		  \texttt{drop :: Int -> [a] -> [a]}
	
	\item Enumerate the cases \\
		  \texttt{drop 0 [] = } \\
		  \texttt{drop 0 (x:xs) = } \\
		  \texttt{drop n [] =}  \\
		  \texttt{drop n (x:xs) = }
		  
	\item Define the simple cases \\
		  \texttt{drop 0 [] = []} \\
		  \texttt{drop 0 (x:xs) = (x:xs)} \\
		  \texttt{drop n [] = []}
		  
	\item Define the remaining cases \\
		  \texttt{drop n (x:xs) = drop (n-1) xs}
		  
	\item Simplify and generalize \\
		  \texttt{drop :: Integral b => b -> [a] -> [a]} \\
		  \texttt{drop 0 xs = xs} \\
		  \texttt{drop \_ [] = []} \\
		  \texttt{drop n (\_:xs) = drop (n-1) xs}
	
\end{enumerate}


\section{Higher order functions}
\textbf{Higher order functions} are functions that return functions as a result. This looks obvious when referring to currying, but with higher order functions we can define also functions that take functions as parameters. 

\begin{lstlisting}[language=haskell]
twice :: (a -> a) -> a -> a
twice f = f . f

Prelude> twice (*2) 3
12

Prelude> twice reverse [1, 2, 3]
[1, 2, 3]
\end{lstlisting}

Higher order functions are useful when defining maps, functions that take as input a function and a list to which elements apply the function. 

\begin{lstlisting}[language=haskell]
map :: (a -> b) -> [a] -> [b]
map f xs = [ fx | x <- xs ]

Prelude> map (+1) [1, 2, 3]
[2, 3, 4]

Prelude> map reverse ["abc", "def"] 
["cba", "fed"]
\end{lstlisting}

Using nested maps it is also possible to work on nested lists

\begin{lstlisting}[language=haskell]
Prelude> map (map (+1)) [[1, 2, 3], [4, 5]]
= {applying outer map}
[map (+1) [1, 2, 3], map (+1) [4, 5]]
= {applying inner maps} 
[[2, 3, 4], [5, 6]]
\end{lstlisting}

Also, with higher order functions it is possible to define filters.

\begin{lstlisting}[language=haskell]
filter :: (a -> Bool) -> [a] -> [a]
filter p xs = [x | x <- xs, p x]

Prelude> filter even [1..10]
[2, 4, 6, 8, 10]
\end{lstlisting}
